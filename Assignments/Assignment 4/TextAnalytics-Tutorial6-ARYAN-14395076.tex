\documentclass{article}
\usepackage{pdfpages}
\usepackage{listings}
\usepackage[T1]{fontenc}
\usepackage{xcolor}
\usepackage{placeins}
\usepackage{float}
\usepackage{underscore}
\usepackage[bookmarks=true]{hyperref}
\usepackage[utf8]{inputenc}
\usepackage[margin=0.5in]{geometry}
\usepackage{graphicx}
\usepackage{subfigure}
\usepackage[english]{babel}
\hypersetup{
	bookmarks=false,    % show bookmarks bar?
%	pdftitle={Text Analytics Tutorial 5},    % title
	pdfauthor={Adam-Ryan},                     % author
	pdfsubject={TeX and LaTeX},                        % subject of the document
	pdfkeywords={TeX, LaTeX, graphics, images}, % list of keywords
	colorlinks=true,       % false: boxed links; true: colored links
	linkcolor=blue,       % color of internal links
	citecolor=black,       % color of links to bibliography
	filecolor=black,        % color of file links
	urlcolor=blue,        % color of external links
	linktoc=page            % only page is linked
}%
\def\myversion{1 }
\date{}
%\title
\usepackage{hyperref}
\begin{document}
	
\section{Exercise 1}
This is the answer to question 1.
\begin{enumerate}
	
	\item Question a
	
	\item Question b
	
	\item Question c
	
\end{enumerate}

\section{Exercise 2}
This is the answer to question 2.
\begin{enumerate}
	
	\item Question a
	
	\item We see that the Jaccard-Distance holds as it's a metric. By applying the Steinhaus transform combined with the symmetric distance of two sets (which is a metric) we arrive at a proof of the triangle quality (the other two properties are trivial).
	
	
	\item Question c
	
\end{enumerate}


\end{document}
